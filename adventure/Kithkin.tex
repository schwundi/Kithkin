\documentclass[10pt,a4paper,onecolumn]{article}
\usepackage[utf8]{inputenc}
\usepackage[headsepline,automark]{scrlayer-scrpage}
\author{Peter Salmen}
\title{Intro}
\ihead{ }
\chead{Btütenstaub Schlaflied}
\ohead{ }
\ifoot{Kingsbaile}
\cfoot{}
\ofoot{Ballyrush}
\begin{document}
\section*{Das Reich der Kithkin}
Die hier wohnenden Menschen gaben dem Reich den Namen Kithkin 
aufgrund der vielen Halblinge die hier wohnen. Kaum jemand kennt 
in diesem Land Halblinge unter einem anderem Namen als Kithkin.
Das Reich wird zwar seit Jahrhunderten von Menschen regiert,
jedoch hat der König stets ein dutzend Kithkin um sich herum, 
die ihm als persönliche Berater dienen. Für Kithkin ist es eine
große Ehre zum Berater des Königs ernannt zu werden.
Auch in den Dörfern und Städten des Landes findet sich diese Gliederung
in sehr ähnlicher Form wieder.

Eine dieser Städte trägt den Namen \textbf{Blütenstaub} und bestand
einst aus zwei Dörfern die nahe beieinander lagen. In \textbf{Kingsbaile}
wohnten überwiegend Menschen und in \textbf{Ballyrush} lebten beinahe
ausschließlich Kithkin. Beide Dörfer wuchsen rasch, dank des guten
Ackerlandes welches beide umgibt. Innerhalb nur weniger Jahre wurden 
so aus einzelnen Häusern eine kleine Stadt. Die vielen bunten Blumen 
die überall wachsen wo kein Ackerbau betrieben wird gaben der Stadt 
ihren wunderschönen Namen. Im Sommer wird die Stadt regelmäßig  mit 
Blütenstaub verziert. Viele nennen dieses Phänomen das 
\textbf{Blütenstaub Schlaflied}. Dem Blütenstaub wird nachgesagt, dass
er die Stadt vor großen Ereignissen bewahrt und diese bis zum Ende der
Zeit friedlich schlafen lässt. Mit dem letztem 
\textbf{Blütenstaub Schlaflied} beginnt die kalte Jahreszeit. Im 
Frühling wird jedes Jahr die Vereinigung der beiden Dörfer und das 
erste \textbf{Blütenstaub Schlaflied} zusammen gefeiert. Die besten Weine
der Region werden extra für dieses Fest produziert. Der berühmteste Wein 
ist der \textbf{Blüstenstaub Festwein}, jedes Jahr ziert eine bunt 
gezeichnete Blumenwiese die Flaschen. Der Wein ist weit über die Grenzen 
des Reiches hinaus bekannt und so reisen Gäste von überall an um diesen 
Wein zu verköstigen. Ein verkauf von mehr als eine Flasche pro Person ist
nicht gestattet. Das Essen des Fests steht dem Wein in nichts nach und 
viele Gerichte werden an den Ständen aufwändig mit essbaren Blumen verziert.
\section*{Kithkin Windreiter}
Viele Erzählungen habt ihr schon gehört über das Fest in 
\textbf{Blütenstaub} und habt nun den Entschluss gefasst euch auf den Weg
zu diesem Fest zu machen. Das Fest wird am kommenden Wochenende von Freitag
bis Montag gefeiert. Ihr denkt, dass ihr fünf Tage brauchen werdet bis ihr
angekommen seid  und so habt ihr nun hastig eure Sachen zusammengepackt 
und euch auf den Weg in dieses kulinarische Abenteuer gemacht.
\newpage
\section*{Zahlen, Daten, Fakten} 
\begin{itemize}
    \item Gespielt wird auf Level 5
    \item Die Hit dice für Level 2-5 werden mit advantage gewürfelt.
    \item Jeder Besitzt ein riding Horse
    \item Jeder besitzt 200 Gold zusätzlich zu seinem Start Equipment
    \item Keiner besitzt zur Zeit einen magischen Gegenstand
    \item Regelwerke    
    \begin{itemize} 
        \item Player's Handbook
        \item Volo's Guide to Monsters
        \item Xanathar's Guide to Everything
        \item Mordenkainen's Tome of Foes
        \item Tasha's Cauldron of Everything
        \item Sword Coast Adventurer's Guide       
    \end{itemize}
    \item Die Attribute werden mit 7*(4d6) bestimmt wie im PhB beschrieben 
        aber mit einem Attributswert mehr als nötig. Die optionale Regel 
        aus Tasha's Cauldron zum erzeugen generischer Rassen ist nicht zugelassen.


\end{itemize}

\end{document}
